\documentclass[12pt]{article}

\usepackage{geometry}
\usepackage{parskip}
\usepackage{tikz}

% This style is used to create block diagrams, you'll find it useful since many of your figures would be of that form, I'll try add more styles in the future :)
\usetikzlibrary{trees,positioning,fit,calc}
\tikzset{block/.style = {draw, fill=blue!20, rectangle,
                         minimum height=3em, minimum width=4em},
        input/.style = {coordinate},
        output/.style = {coordinate}
}

\usepackage[section]{minted}
\usepackage{xcolor}
\usemintedstyle{porland}

\usepackage{chngcntr}
\counterwithin{figure}{section}

\usepackage{tocbasic}
\setuptoc{lol}{levelup}

% \usepackage{indentfirst}
\geometry{a4paper, margin=1in}

%----------EDIT COVER INFO HERE -----------------%

\def \LOGOPATH {assets/logo-usth.png}
\def \DEPARTEMENT {Department of Information and Communication Technology}
\def \COURSENUM {ENCS101}
\def \COURSENAME {Network Programming}
\def \REPORTTITLE {C Program to resolve domain names}
\def \STUDENTNAME {NGUYEN Hoang Minh}
\def \STUDENTID {BI11-181}
\def \INSTRUCTOR {TRAN Giang Son}

%------------------------------------------------%

\setlength{\parindent}{2em}
\setlength{\parskip}{0em}

\begin{document}

\pagenumbering{Roman}

\begin{titlepage}
    \vfill
    \begin{center}
        \includegraphics[width=0.7\textwidth]{\LOGOPATH} \\
        \hfill \\
        \Large{\DEPARTEMENT} \\
        \Large{\COURSENAME} \\
        \vfill
        \textbf{\LARGE{\REPORTTITLE}}
    \end{center}
    \vfill
    \begin{flushleft}
        \Large{\textbf{Prepared by:} \STUDENTNAME\;-\;\STUDENTID} \\
        \Large{\textbf{Instructor:} \INSTRUCTOR} \\
        \Large{\textbf{Date:} \today}
    \end{flushleft}
    \vfill
\end{titlepage}


\tableofcontents
\clearpage

\setlength{\parskip}{\baselineskip}%

\pagenumbering{arabic}

%--------------INTRODUCTION ---------------------%

\section{Problem Analysis}

\subsection{Get domain name from user}
\begin{itemize}
    \item Input domain name from CLI arguments
    \item Input domain name from user
\end{itemize}

\subsection{Resolve domain name}
\begin{itemize}
    \item Hostname is not found
    \item Hostname is resolved successfully with multiple IPs (if possible)
\end{itemize}

\section{Get domain name from user}
Initially, a char array of length 255 is declared to store the domain name
\begin{verbatim}
    char input[255];
\end{verbatim}

\subsection{Domain name comes from program's arguments}
\Verb"argc" is the argument count. If \Verb"argc" is greater than one, there is at least one argument. We copy the second argument of \Verb"argv" (the first argument is the program name) into the \Verb"domain" char array
\begin{verbatim}
    if (argc > 1)
    {
        strcpy(input, argv[1]);
    }\end{verbatim}

\subsection{Domain name comes from user input}
If no argument is provided, we prompt the user to enter a domain name and store the input into the \Verb"domain" char array
\begin{verbatim}
    else
    {
        printf("Enter a domain name: ");
        fgets(input, sizeof(input), stdin);
        input[strlen(input) - 1] = '\0';
    }
\end{verbatim}
\clearpage

\section{Resolve domain name}
Utilizing the function \Verb"gethostbyname()", we can resolve the \Verb"domain" and obtain either a pointer to a \Verb"struct hostent" or a null pointer.
\subsection{Hostname is not found}
If we obtain a null pointer, the domain name cannot be resolved. Then we inform the user and exit the program
\begin{verbatim}
    if (host_info == NULL)
    {
        printf("No such host\n");
        exit(1);
    }
\end{verbatim}
\subsection{Hostname is resolved successfully}
If we obtain a pointer to a \Verb"struct hostent", we loop through its address list, convert each address to a char array using the function \Verb"inet_ntop̣()" and print the result 
\begin{verbatim}
    for (int i = 0; host_info->h_addr_list[i] != NULL; i++)
    {
        struct in_addr *address;
        address = (struct in_addr *)(host_info->h_addr_list[i]);

        printf("%s has address %s\n", input, inet_ntoa(*address));
    }
    
\end{verbatim}


\clearpage
\section{Demonstration}
Compile the program to an output file name \Verb"mynslookup"
\subsection{Run the program locally}
\begin{verbatim}
> ./mynslookup stackoverflow1.com
No such host

> ./mynslookup
Enter a domain name: facebook.com
facebook.com has address 31.13.75.35


> ./mynslookup stackoverflow.com
stackoverflow.com has address 151.101.1.69
stackoverflow.com has address 151.101.193.69
stackoverflow.com has address 151.101.129.69
stackoverflow.com has address 151.101.65.69

\end{verbatim}
\subsection{Run the program on a VPS in Singapore}
\begin{verbatim}
> ./mynslookup stackoverflow1.com
No such host

> ./mynslookup
Enter a domain name: facebook.com
facebook.com has address 157.240.235.35


> ./mynslookup stackoverflow.com
stackoverflow.com has address 151.101.1.69
stackoverflow.com has address 151.101.193.69
stackoverflow.com has address 151.101.129.69
stackoverflow.com has address 151.101.65.69

\end{verbatim}
The resolved IP address of facebook.com on my local machine \Verb"31.13.75.35" is different from the one from my VPS \Verb"157.240.235.35" because one domain name can map to multiple IP addresses and the geological distance can cause the mapping to behave differently in diffent region.
\clearpage


\end{document}